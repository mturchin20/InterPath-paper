\documentclass[12pt, a4paper]{article}
\usepackage[english]{babel}
\usepackage[utf8x]{inputenc}
\usepackage[T1]{fontenc}
\usepackage[a4paper]{geometry}
\usepackage{amsmath}
\usepackage{graphicx}
\usepackage[colorlinks=true, allcolors=blue]{hyperref}
\usepackage{epsfig,amsfonts}
\usepackage{natbib}
\usepackage{amssymb}
\usepackage{amsthm}
\usepackage{authblk}
\usepackage{setspace}
\usepackage{hypcap}

\title{Supplementary: Differential complex trait architecture across humans: epistasis identified in non-European populations at multiple genomic scales}
\author[1,2]{Michael C. Turchin}
\author[1,3]{Isabella Ting}
\author[1,4,5,*]{Lorin Crawford}
\author[1,2,*,$\dag$]{Sohini Ramachandran}
\affil[1]{Center for Computational Molecular Biology, Brown University}
\affil[2]{Department of Ecology and Evolutionary Biology, Brown University}
\affil[3]{Department of Computer Science, Brown University}
\affil[4]{Department of Biostatistics, Brown University}
\affil[5]{Center for Statistical Science, Brown University}
\affil[$\ast$]{indicates these authors contributed equally}
\affil[$^\dag$]{To whom correspondence should be addressed: sramachandran@brown.edu}

\begin{document}

\maketitle

\section{Supplementary Note}\label{Supplementary Note}

\subsection{Population Subset Quality Control}

We then conducted standard quality control (QC) procedures on each of these population subsets. Note that we focused our analyzes on the genotyped chip data throughout the project. First we conducted SNP-level QC by dropping variants that did not meet the following criteria:  minor allele frequency (MAF) >= .01, genotype missingness <= 5\%, and Hardy-Weinberg equilibrium test p-value >= $1\times10^{-6}$. We then conducted individual-level QC via the following steps. Individuals were removed if they did not have genotype missingness >= 5\%. Individuals were also removed if they were a 3\textsuperscript{rd} degree relative or more to someone else in the dataset; specifically the KING relatedness values provided with the UKB data were used to identify related individuals, and one individual from every pair of 3\textsuperscript{rd} degree or more relatives was removed. Individuals were also dropped if they were tagged by any of the following three flags from the UKB data: `het.missing.outliers', `putative.sex.chromosome.aneuploidy', and `excess.relatives'. Lastly, individuals were removed if they were determined to be PCA outliers; this was conducted by running FlashPCA (version 2.1) \citep{Abraham2017} in R on each population subset separately and identifying individuals that had PC values greater than 7 standard deviations away from the mean for any of the top 6 PCs. 

After this first round of QC procedures, we then proceeded to impute our current population subsets. Since most of the analyses in this project utilized genetic relatedness matrices (GRMs), and variants need to have no missing data for these GRMs, we used imputation primarily to maximize the number of genotyped SNPs that would not be dropped by this stringent threshold (as opposed to using imputation to increase the number of SNPs we were analyzing). To conduct this imputation, we uploaded our population subsets to the University of Michigan Imputation Server \citep{Das2016} and used the following options: Minimac3 for the imputation software, 1000G phase 3 v5 for the reference panel, and Eagle v2.3 for the phasing software. Completed imputed files were then downloaded from the Imputation Server afterwards and treated to further QC steps: imputed variants were intersected back to the original set of genotyped chip variants, variants with imputation quality scores < .3 were removed, and variants that had genotype missingness rates > 0\% were also removed. These steps represent the last of our QC and imputation procedures, and information on the final forms of our UKB population subsets can be found in Supplementary Table (table).

\begingroup
\bibliographystyle{apalike}
\setstretch{1.0}
%\bibliography{Supplemental}
\endgroup

\end{document}
